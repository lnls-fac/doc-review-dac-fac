% Latex template: mahmoud.s.fahmy@students.kasralainy.edu.eg
% For more details: https://www.sharelatex.com/learn/Beamer

\documentclass{beamer}					% Document class

\usepackage[portuguese]{babel}				% Set language
\usepackage[utf8x]{inputenc}			% Set encoding

\mode<presentation>						% Set options
{
  \usetheme{default}					% Set theme
  \usecolortheme{default} 				% Set colors
  \usefonttheme{default}  				% Set font theme
  \setbeamertemplate{caption}[numbered]	% Set caption to be numbered
}

\setbeamertemplate{navigation symbols}{}
\setbeamertemplate{footline}[frame number]
\setbeamercovered{transparent}

% Uncomment this to have the outline at the beginning of each section highlighted.
%\AtBeginSection[]
%{
%  \begin{frame}{Outline}
%    \tableofcontents[currentsection]
%  \end{frame}
%}

\usepackage{graphicx}					% For including figures
\usepackage{booktabs}					% For table rules
\usepackage{hyperref}					% For cross-referencing

\title{Revisão de Atividades da FAC}	% Presentation title
%\author{Author One}								% Presentation author
\institute{LNLS.DAC.FAC}					% Author affiliation
\date{2023-09-25 -- 2023-10-06}									% Today's date	

\begin{document}

% Title page
% This page includes the informations defined earlier including title, author/s, affiliation/s and the date
\begin{frame}
  \titlepage
  \href{https://github.com/lnls-fac/doc-review-dac-fac}{\beamergotobutton{Link para o repo github desta apresentação: https://github.com/lnls-fac/doc-review-dac-fac}}
\end{frame}

% Outline
% This page includes the outline (Table of content) of the presentation. All sections and subsections will appear in the outline by default.
\begin{frame}{Outline}
  \tableofcontents
\end{frame}

% The following is the most frequently used slide types in beamer
% The slide structure is as follows:
%
%\begin{frame}{<slide-title>}
%	<content>
%\end{frame}


\section{Estudos de máquina - 25/09 Instabilidade CB}

\begin{frame}{25/09 Instabilidade CB}
    \begin{itemize}
		\item Questão: troca de eletrônica LLRF introduziu uma maior oscilação longitudinal.
        \item Solução: mudança da freq. síncrotron para long de múltiplos de "64" Hz, através de mudança da tensão de aceleração da P7.
	\end{itemize}
\end{frame}



\section{Estudos de máquina - 02/10 Otimização LLRF da P7Cav}

\begin{frame}{02/10 Otimização LLRF da P7Cav}
    \begin{itemize}
		\item Reconfiguração de parâmetros PID do LLRF da P7Cav.
	\end{itemize}
	% \input{tables/table1.tex}
\end{frame}


% \begin{frame}{02/10 Otimização LLRF da P7Cav}
% 	\begin{figure}[H]
% 		\centering
%         \includegraphics[width=.5\textwidth]{figures/figure1.png}
%         \caption{Caption for figure one.}
%         \label{fig:figure1}
% 	\end{figure}
% \end{frame}

% \begin{frame}{02/10 Otimização LLRF da P7}

% \begin{columns}

% \begin{column}{0.5\textwidth}
% 	\begin{figure}[H]
% 		\centering
%         \includegraphics[width=.9\textwidth]{figures/figure1.png}
%         \caption{Caption for figure one.}
%         \label{fig:figure1}
% 	\end{figure}
%  \end{column}

%  \begin{column}{0.5\textwidth}
% 	\begin{figure}[H]
% 		\centering
%         \includegraphics[width=.9\textwidth]{figures/figure1.png}
%         \caption{Caption for figure two.}
%         \label{fig:figure1}
% 	\end{figure}
%  \end{column}
 
% \end{columns}
% \end{frame}

% \begin{frame}{Slide with references}
% \begin{itemize}
%     \item<1->This is to reference a figure (Figure \ref{fig:figure1})\\
%     \item<2->This is to reference a table (Table \ref{tab:table1})\\
%     \item<3->This is to cite an article \cite{Ahmed2018a}\\
%     \item<4->This is to add an article to the references without mentioning in the text \nocite{Ahmed2018a}\\
% \end{itemize}
% \end{frame}

\section{Estudos de máquina - 03/10 Ajuste da função dispersão}

\begin{frame}{03/10 Ajuste da função dispersão}
    \begin{itemize}
		\item Usando a matrix resposta $\delta \eta_y / \delta LK_s$ atuamos nos skew quads e ajustamos a função dispersão para a nominal (mínima).
        \item Medimos acomplamento global $\sim$ 4.6\% e corrigimos com os botões de acoplamento tradicionais para $\sim$ 1.4\%.
	\end{itemize}
\end{frame}

% \begin{frame}{Estudos de máquina - 25/09 Instabilidade CB}
%     \begin{itemize}
% 		\item<1-> Questão: troca de eletrônica LLRF introduziu uma maior oscilação longitudinal.
%         \item<2-> Solução: mudança da freq. síncrotron para long de múltiplos de "64" Hz, através de mudança da tensão de aceleração da P7.
% 	\end{itemize}
% \end{frame}

% \begin{frame}{Estudos de máquina - 02/10 Otimização LLRF da P7}
% 	\begin{columns}
% 		\column{.5\textwidth}
%         Text goes in first column.
        
%         \column{.5\textwidth}
%         Text goes in second column
% 	\end{columns}
% \end{frame}

\section{Atividades - Calibração modelo RADIA do DELTA52}

\begin{frame}{Calibração modelo RADIA do DELTA52}
    \begin{itemize}
		\item A ideia é ter um modelo RADIA 3D do ID que explique as medidas de mapas de campo feitas com sensor Hall, que no caso do DELTA, só podem ser feitas no plano $y = 0$. Com este modelo calibrado podemos fazer RK para resolver a traj. 3D e obter os mapas de kicks transversais $x'(x_0, y_0)$ e $y'(x_0, y_0)$ para estudar o efeito do ID na ótica e abertura dinâmica.
        \item Usando as medidas de mapas de campo com sensor Hall, calibramos o modelo RADIA do ID (sem magic finger)
	\end{itemize}
\end{frame}

\section{References}

% Adding the option 'allowframebreaks' allows the contents of the slide to be expanded in more than one slide.
% \begin{frame}[allowframebreaks]{References}
% 	\tiny\bibliography{2023-10-06/references}
% 	\bibliographystyle{apalike}
% \end{frame}

\end{document}
